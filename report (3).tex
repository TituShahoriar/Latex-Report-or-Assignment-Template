\documentclass{article}

% Packages
\usepackage[utf8]{inputenc}
\usepackage[T1]{fontenc}
\usepackage{lmodern}
\usepackage{graphicx}
\usepackage{xcolor}
\usepackage{lipsum} % For dummy text, you can remove this package
\usepackage{tikz}
\usepackage{titlesec}
\usepackage{hyperref}

% Custom Color Scheme
\definecolor{primarycolor}{RGB}{26, 115, 232}
\definecolor{secondarycolor}{RGB}{231, 76, 60}
\definecolor{accentcolor}{RGB}{0, 166, 90}

% Section Formatting
\titleformat{\section}
{\color{primarycolor}\normalfont\Large\bfseries}
{\thesection}{1em}{}

% Title Page
\title{
    \vspace{-3cm}
    \begin{tikzpicture}[remember picture,overlay]
        \node[anchor=north east,inner sep=0pt] at (current page.north east)
        {\includegraphics[width=4cm]{university_logo.png}};
    \end{tikzpicture}\\
    {\color{primarycolor}\Huge Your Report Title}
}
\author{Your Name}
\date{\today}

\begin{document}

\maketitle

% Table of Contents
\tableofcontents
\newpage

% Sections
\section{Introduction}
\lipsum[1-2] % Replace with your actual content

\section{Methodology}
\lipsum[3] % Replace with your actual content

\subsection{Data Collection}
\lipsum[4] % Replace with your actual content

\subsection{Data Analysis}
\lipsum[5] % Replace with your actual content

% Figures
\section{Results}
\lipsum[6] % Replace with your actual content

\begin{figure}[htb]
    \centering
    \includegraphics[width=0.8\textwidth]{result_plot.png}
    \caption{Result plot}
    \label{fig:result}
\end{figure}

Figure \ref{fig:result} shows the result plot obtained from the experiment.

% Tables
\section{Discussion}
\lipsum[7] % Replace with your actual content

\begin{table}[htb]
    \centering
    \begin{tabular}{|c|c|}
        \hline
        \textbf{Parameter} & \textbf{Value} \\
        \hline
        A & 10 \\
        B & 20 \\
        \hline
    \end{tabular}
    \caption{Experimental Parameters}
    \label{table:params}
\end{table}

Table \ref{table:params} lists the experimental parameters used in the study.

% Equations
\section{Conclusion}
\lipsum[8] % Replace with your actual content

The equation for calculating the energy is given by:

\begin{equation}
    E = mc^2
    \label{eq:energy}
\end{equation}

Equation \ref{eq:energy} represents the famous mass-energy equivalence.

% References
\begin{thebibliography}{9}
    \bibitem{ref1} Author. (Year). Title of the article. \textit{Journal Name}, \textit{Volume}(Issue), Page numbers. DOI/URL.
    % Add more references as needed
\end{thebibliography}

\end{document}
